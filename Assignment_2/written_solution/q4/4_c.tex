There are two insights that we conclude from the results. First, the RNN seems to be better at predicting the next token than the Bi-Gram model for both of the datasets, as we can see from its much lower perplexity values.
Second, the perplexity of the RNN is lower for the Shakespeare dataset than for the Wikipedia dataset.
The reason that the RNN is much better than the Bi-Gram model is, in my opinion, that the RNN is charachter level model. 
That means that in each step it has to predict the next token out of 26 possible tokens (the 26 letters in the English alphabet). 
On the other hand, the Bi-Gram model is a word level model, and therefore it has to predict the next token out of a much larger vocabulary. Hence, it's not a "fair game", in that regard.

Another reason that may contribute to the superiority of the RNN is that the RNN is not a bigram model, i.e each character prediction depends (theoretically, at least) on all the previous characters, while the Bi-Gram model only depends on the previous character. 
We've seen in class (and understand intuitively) that the markovian assumption that underlies the bigram model does not hold in reality.

Finally, the reason that the RNN is better for Shakespeare than for Wikipedia is that the RNN was trained on Shakespeare, and therefore it is more familiar with the Shakespeare style of writing.