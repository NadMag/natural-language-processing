\newenvironment{simplechar}{%
   \catcode`\$=12
   \catcode`\&=12
   \catcode`\#=12
   \catcode`\^=12
   \catcode`\_=12
   \catcode`\~=12
   \catcode`\%=12
   \catcode`\<=12
   \catcode`\>=12
}{}
\begin{simplechar}
Example 1:
This is the latest entry in the long series of films with the French agent, O.S.S. 117 (the French answer to James Bond). The series was launched in the early 1950's, and spawned at least eight films (none of which was ever released in the U.S.). 'O.S.S.117:Cairo,Nest Of Spies' is a breezy little comedy that should not...repeat NOT, be taken too seriously. Our protagonist finds himself in the middle of a spy chase in Egypt (with Morroco doing stand in for Egypt) to find out about a long lost friend. What follows is the standard James Bond/Inspector Cloussou kind of antics. Although our man is something of an overt xenophobe,sexist,homophobe, it's treated as pure farce (as I said, don't take it too seriously). Although there is a bit of rough language & cartoon violence, it's basically okay for older kids (ages 12 & up). As previously stated in the subject line, just sit back,pass the popcorn & just enjoy.

Explanation:
The sentence "breezy little comedy that should not...repeat NOT, be taken too seriously" has negative connotation, it can potentialy make the model classify this example as negative.

Example 2:
I was expecting "Born to Kill" to be an exciting, high-tension film noir. Instead, it's got two good action set-pieces (one at the beginning and one at the end) and some marvelously atmospheric cinematography by Robert de Grasse (usually a "glamor" cameraman and a surprising credit for a noir), but the rest of the film is pretty boring. Lawrence Tierney goes through his psycho kick but it's a strictly by-the-numbers performance, mechanically churning out what the audience expected from him after "Dillinger" (an overrated movie but at least better than this). Claire Trevor's character is too stupid and unmotivated to have any audience appeal, and the action (such as it is) stays so resolutely inside that damned house in San Francisco the film becomes claustrophobic instead of genuinely thrilling. It's one of those movies in which the supporting players -- notably Elisha Cook, Jr. (whose character's homoerotic itch for Tierney's is one of the few subtleties in an otherwise pretty obvious script) and Isabel Jewell -- out-act the leads. It also doesn't help that, nearly half a century after Alfred Hitchcock and Anthony Perkins revolutionized the depiction of psycho killers on the screen in "Psycho," Tierney's is so gross and obvious he might as well have "PSYCHO" tattooed on his forehead. Also, there's no indication in the film as it stands as to why the source novel was called "Deadlier than the Male" -- but perhaps James Gunn made the female characters stronger and more interesting than they are in the film. "Born to Kill" is a real disappointment from Robert Wise, who already had some quality movies under his belt and would go on to a stellar career.

Explanation: 
Although the review is obviously negative, it contains some positive descriptions that might cause a false classification.  "exciting", "noir", "good action set-piece", "marvelously atmospheric cinematography", are all examples for the positive words contained in this review.

Example 3:
It seems everyone wants to jump on the bandwagon and say "Maha Go Go Go"....The word is MACHA........Like "Mach".....Pronounced maa - ka"...<br /><br />I grew up with this series in the early 70's here in LA on the late and VERY lamented channel 56...Before that there was Tetsuwan Atomu (Astro Boy), dating from 1963 on ol' KHJ TV. Astro Boy was the first TV example of anime we got here in the states...I was into anime as a kid and followed it until the late 80's when, by then it'd become a series of badly animated "talking heads", a phenomenon which has only gotten worse. 'Nuff said.<br /><br />As for "Speed Racer", I really enjoyed the basics there, the POV shots, the cinematic aspects of live action skillfully adopted to animation...That was fairly typical of most Japanese anime back then...Graphics graphics graphics! Take note sometime how obviously the series was inspired by Stanley Kramer's film "Grand Prix" (1966), especially the redone American credits....<br /><br />Oh yeah, I have the original comics from which the series is based, so I know of which I speak.<br /><br />What were we doing animation-wise besides crap like Johnny Quest?.....Th' same ol' stuff we'd been doin' since the 20's....Ho-hum!<br /><br />I guess the real problem I had/have with the way anime was/is shown on American TV is the hatchet job done on the scripts, credits, etc to "sanitize" them for American audiences...I won't go into other programs as we're talking' Speed here.<br /><br />Look at clowns like Peter Fernandez as one of the culprits here, as he was 99% responsible for the re-writes of the series...Not to mention the voice of Speed, Racer X and others...Between him and the goofs at Trans/Lux ( Think Felix the Cat and the Mighty Hercules - oy vey!) they took a slick, very sophisticated show and dropped it down to the level of Sesame Street. Think "Cruncher Bloch", The "Forthebird Company", "Skull Duggery"...If I go on I'll puke.<br /><br />This series dates from 40-odd years ago but I, at the time, was keen enough to feel insulted by the dumbing down of this and other Japanese programs...I mean it's obvious when someone's getting' killed but they either remove it or gloss it over........Pleeeeeze!<br /><br />Good show - originally. Sadly all the more recent incarnations of the series have that CRAPPY "made in Korea" look, not to mention being nauseatingly "pc" in content. Even the Japanese outsource their animation now..<br /><br />Try watchin' the original Japanese opening on YouTube sometime...It sends chills up my spine.....If only......Oh well. Robert

Explanation:
The review is too longs and contains a lot of information regarding unrelated anime films. Combination of long text together with many walk-arounds before getting the point can confuse our model.

Example 4:
Recap: Since the warrior queen Gedren raised and slaughtered most of Sonja's family, she has trained in the art of sword fighting. Now, Gedren has taken a very powerful talisman, that threatens to destroy the world if not destroyed, killing Sonja's sister in the process. Now Sonja is out for revenge, and to save the world. Along the way, she meets the very Conan-like (but not Conan, no!) Kalidor, the child-prince Tarn and his bodyguard Falcon. At first Sonja declines all help, but is later forced to accept it, and together they go to save the world.<br /><br />Comments: When you watch a movie like this, and you think that it is the story that the is the best element in this movie, the movie is in big trouble. Because 1) a movie like this should draw its strength upon good swordfights and effects, and 2) the story is really, really bad. It is simple, and uncomplicated and really offers nothing in way of character development or even suspense. It is predictable and boring, and the obvious couple, Sonja and Kalidor, has no chemistry at all. And the kid is just annoying. And most of the scenes is drawn out so long that they become boring. Though the movie is not very long, it has not material enough to fill its time. And so back to point 1). The fighting is slow, uneventful and really bad. It clearly shows most fighters clearly blocking the opponents strokes far ahead of the opponent has even begun to strike. In my honest opinion, I believe most kids, fighting with sticks, creates more exciting fights playing knights than this movie did. All in all, this is a really bad spin-off, that should be avoided by all who liked the Conan-movies.<br /><br />2/10

Explanation:
While the review itself is negative, it begins with a long recap of the storyline that contains positive words that can confuse the model. For example: "art of sword fighting", "very powerful". In addition, the sentence "the story is the best element" that is written later, might confuse the model.

Example 5:
Not as bad as some people say...This is a unofficial Bond movie and a remake of "Thunderball", written by Kevin McClory (co- producer in "Thunderball"). Well, the cast is very very interesting, Maria Brandauer is a great Bond- villain, Kim Basinger and Barbara Carrera are just like the "original" Bond- girls, plus Rowan Atkinson and a truly great Edward Fox, who looks really refreshing in the "M" role. In fact, the whole movie is refreshing and gives some new impulses. Sean Connery does it once more confident and charming, except that he looks a little bit too old. But alright, he is the original Bond and it was great to see him once more in this role. The locations are also typical- Bahamas, France, etc. The only thing that really fails is the music score, the song "Never say never again" is O.K., but the theme song is just missing. All in one, a nice try to make a difference from the comic and silly Roger Moore movies like "Moonraker". Only if there was another story, "Thunderball" was a excellent movie and really did not needed a remake

Explanation:
The review itself is ambiguous and shows some both negative and positive aspects of the film.

\end{simplechar}