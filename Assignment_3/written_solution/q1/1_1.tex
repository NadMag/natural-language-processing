\subsubsection*{(a)}
Let us note that due to the formula that defines each $\alpha_i$ the sum of all $\alpha_i$ is 1, and each value is positive. 
Hence, by looking at each value as category we easily get that this is a categorical distribution. \newline
\subsubsection*{(b)}
This would happen if 
\begin{enumerate}
    \item there exists some $k_j$ where $k_j^Tq$ is large and
    \item for all other $k_i$ where $i \neq j$ $k_i^Tq$ is small.
\end{enumerate} 
Let us remember that $a^Tb = |a||b|\cos(\theta)$ where $\theta$ is the angle between $a$ and $b$. Hence for the conditions above to hold we need that $k_j$ is parallel to $q$ and all other $k_i$ are in the opposite direction to $q$. \newline
Plus, we need that either $q$ is very large (i.e large norm) or that $k_j$ is very large or that the other $k_i$ vectors norm is very large. \newline
\subsubsection*{(c)}
Assuming that the conditions above are correct, we get that $c \approx v_j$ \newline
\subsubsection*{(d)}
This means that the output $c$ will depend almost only on a single $v_j$ that corresponds to the $k_j$ that is parallel to $q$. 
This is, obviously, not desirable since we want the output to be contextualized, i.e depend on the full sentence. Without this property the model will not be expressive enough.