\subsubsection*{(a)}
Let us define the following matrix $M$:
$$M=c_1 \frac{a_1 a_1^T}{a_1^T a_1} + c_2 \frac{a_2 a_2^T}{a_2^T a_2} + \dots + c_m \frac{a_m a_m^T}{a_m^T a_m}$$
If we multiply $M$ by $a_i$ we get that 
$$Ma_i = c_i a_i$$
i.e, $a_i$ is an eigenvector of $M$ with eigenvalue $c_i$. That means that 
$$Mv_a = M(a_1+ \dots + a_m) = Ma_1+ \dots + Ma_m=c_1a_1+ \dots +c_ma_m=v_a$$
Plus, when multiplying $M$ by $v_b$ we get that
$$Mv_b = c_1 \frac{a_1 a_1^T v_b}{a_1^T a_1} + c_2 \frac{a_2 a_2^T v_b}{a_2^T a_2} + \dots + c_m \frac{a_m a_m^T v_b}{a_m^T a_m}$$
we know that $a_i^T v_b = 0$ for all $i$ since 
$$a_i^T v_b = a_i^T b_1 + ... + a_i^T b_m = 0$$
which is true because the two basis are supposed to be orthogonal. Hence
$$Mv_b = 0$$
Hence we finally get that $$Ms=Mv_a+Mv_b=v_a$$ \newline
\subsubsection*{(b)}
we define $q$ as follows:
$$q = (k_a + k_b)*H$$ where $H$ is some Huge number. It holds that 
$$k_a q = k_b q = H$$
Plus, it holds that 
$$ k_i q = 0 $$ if $i \neq a,b$. We get that $\alpha_a = \alpha_b = \frac{e^H}{n-2 + 2e^H}$ and $\alpha_i=\frac{1}{n-2 + 2e^H}$ for $i \neq a,b$. \newline
For large enough $H$ we get that $\alpha_a \approx \alpha_b \approx \frac{1}{2}$ and $\alpha_i \approx 0$ for $i \neq a,b$. Hence $c \approx \frac{1}{2}(v_a + v_b)$