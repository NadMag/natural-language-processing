\documentclass{article}
\usepackage[utf8]{inputenc}
\usepackage{fancyhdr}
\usepackage{lastpage}
\usepackage{amsfonts}
\usepackage{amsmath}
\usepackage{amssymb}
\usepackage{bm}

\usepackage[shortlabels]{enumitem}
\usepackage[noabbrev, capitalise]{cleveref}

\usepackage{geometry}
 \geometry{
 a4paper,
 top=20mm,
 bottom=25mm,
 left=25mm,
 right=25mm,
 }

% define your IDs here:
\newcommand{\firststudentid}{123456789}
\newcommand{\secondstudentid}{987654321}

\pagestyle{fancy}
\fancyhf{}
\rhead{Written Solution for Assignment 1}
\chead{\firststudentid \qquad \secondstudentid}
\lhead{Natural Language Processing}
\rfoot{Page \thepage \hspace{1pt} of \pageref{LastPage}}
\renewcommand{\footrulewidth}{1pt}
 
\setlength{\parindent}{0pt}
\setlength{\parskip}{1em}
\renewcommand{\baselinestretch}{1.25}

\renewcommand{\thesubsection}{\thesection.\alph{subsection}}
\renewcommand{\thesubsubsection}{\thesubsection.\roman{subsubsection}}

\begin{document}

\refstepcounter{section}
\section{Understanding word2vec}
\subsection{}
YOUR ANSWER HERE
\subsection{}
YOUR ANSWER HERE
\subsection{}
YOUR ANSWER HERE
\subsection{}
YOUR ANSWER HERE
\subsection{}
YOUR ANSWER HERE
\refstepcounter{subsection}
\subsubsection{}
YOUR ANSWER HERE
\subsubsection{}
YOUR ANSWER HERE
\subsubsection{}
YOUR ANSWER HERE
\section{Implementing word2vec}
\setcounter{subsection}{4}
\section{Paraphrase Detection (theoretical)}
\subsection{}
We note that because of the relu all of the elements in both $relu(x_1)$ and $relu(x_2)$ are positive, 
because of that the dot product $relu(x_1)^T relu(x_2)$ is composed of a sum of positive arguments, hence 
$$relu(x_1)^T relu(x_2) > 0.5$$
Now, we note that for every positive scalar $i$, it holds that $\sigma (i) > 0.5$. Hence, we arrive at the conclusion that the probability for every sample is
$$p(Paraphrases|x_1, x_2) = \sigma (relu(x_1)^T relu(x_2)) > 0.5$$
Assuming that our decision threshold is $0.5$ it holds that we predict True for every pair.
Hence, we predict correctly for True pairs and wrong for False pairs. Our accuracy is 
$$\frac{1}{1+3}=25 \% $$
\subsection{}


\section{Paraphrase Detection (theoretical)}
\subsection{}
We note that because of the relu all of the elements in both $relu(x_1)$ and $relu(x_2)$ are positive, 
because of that the dot product $relu(x_1)^T relu(x_2)$ is composed of a sum of positive arguments, hence 
$$relu(x_1)^T relu(x_2) > 0.5$$
Now, we note that for every positive scalar $i$, it holds that $\sigma (i) > 0.5$. Hence, we arrive at the conclusion that the probability for every sample is
$$p(Paraphrases|x_1, x_2) = \sigma (relu(x_1)^T relu(x_2)) > 0.5$$
Assuming that our decision threshold is $0.5$ it holds that we predict True for every pair.
Hence, we predict correctly for True pairs and wrong for False pairs. Our accuracy is 
$$\frac{1}{1+3}=25 \% $$
\subsection{}
If we have to keep the relu functions at the last layer, then we can map the sigmoid into $[0,1]$ by adding this computation
$$2*(\sigma (relu(x_1)^T relu(x_2)) - 0.5)$$
(This is equivalent to changing the threshold from 0.5 to 0.75.)
\end{document}
